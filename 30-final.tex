\section{Классификация существующих методов определения заимствований в исходных кодах программ}

На основе~\cite{treb} была дана характеристика точности на небольшой объеме исходного кода. Все остальные характеристики оценивались на основе самых распространенных алгоритмических реализаций из каждой группы методов.

В таблице \ref{tab:compare} приведена классификация ранее рассмотренных методов определения заимствований в исходных кодах программ. По горизонтали расположены критерии сравнения методов определения плагиата.
\begin{enumerate}
	\item Поддержка нескольких языков программирования.
	\item Определение типов заимствований.
	\begin{enumerate}
		\item[2.1)] Изменение комментариев и добавление пустых строк.
		\item[2.2)] Изменение имен функций, переменных и типов данных.
		\item[2.3)] Изменение порядка выполнения кода.
		\item[2.4)] Выделение частей кода в функции.
	\end{enumerate}
	\item Точность на небольшом объеме исходного кода. Находится как среднее из точности определения каждого из типа заимствований. 
	\item Низкоуровневое сравнение кода (ассемблерный листинг или байткод).
\end{enumerate}


\begin{table}[hbtp]
	\begin{center}
		\begin{flushleft}
			\caption{\label{tab:compare}Классификация существующих методов определения заимствований в исходных кодах программ}
		\end{flushleft}
		\begin{tabular}{|l | l | l | l | l | l | l | l |} 
			\hline 
             ~					& {1}	 &    {2.1} &   {2.2} & {2.3} & {2.4} & {3} & {4} \\ \hline
			Текст  & Нет  &   100\%  & 80\%    &  76\%   &  65\%    &  \textbf{80\% }& Нет  \\ \hline
			Токены  &  Да &   100\%  &  92\%   & 87\%  &  74\%  &    \textbf{88\% } & Нет  \\ \hline
		    Метрики &   Да &   100\%   &  87\%   & 95\%    &  72\%   & \textbf{89\% }& Нет  \\ \hline
			Деревья   &   Да &    100\%  &  93\%   &  91\%   &  60\%  &  \textbf{84\%}& Нет  \\ \hline
			Низкоуровневый код &  Да &    99\%  &  100\%   &   85\%   &   80\%  &  \textbf{91\%}& Да \\  \hline
		\end{tabular}
	\end{center}
\end{table}

\pagebreak
Выявлено, что алгоритмы сравнения текста дают высокую точность при определении плагиата~$\approx{100\%}$, когда исходный код был видоизменен только с помощью комментариев и пустых строк . Такой метод применим в сферах, когда исходный код может быть точной копией других кодов. Например, если один студент скопировал работу другого студента и поленился что-то менять. Также его можно использовать в качестве первичной проверки, так как он работает быстрее других алгоритмов. 

Подход сравнения токенов показал высокую точность при изменении как комментариев, так и переменных или типов данных. Его также можно использовать в сфере образования, например, если все работы, которые необходимо проверить, написаны на одном языке программирования и реализуют одинаковый функционал. В таком случае средняя точность $\approx{88\%}$.

Алгоритмы, использующие сравнение метрик аналогичны токенам. Но они дают более высокую точность~$\approx{95\%}$ (у токенов~$\approx{87\%}$) при изменении порядка выполнения кода. Такой подход применим, когда исходные коды разных людей решают одинаковую задачу и добавление пустых или лишних циклов недопустимо.

Сравнение синтаксических деревьев -- единственный алгоритм, которые дает точность более~$\approx{90\%}$ на первых трех типах заимствований. Он хорошо поддерживает несколько языков программирования и в зависимости от выбранного алгоритма сравнения двух деревьев может применяться для разных задач. Хорошей областью применимости являются соревнования по программированию, когда участники решают одинаковые задачи, но на разных языках.

Низкоуровневое сравнение кода показало самую высокую среднюю точность~$\approx{91\%}$, это связано с тем, что этот алгоритм находит модификации четвертого типа.  Такой подход применим, когда все программы одинаково компилируются и имеют заведомо оговоренных типы данных, например, в курсе программирования на Си c автоматическим тестированием.