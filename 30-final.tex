\section{Классификация существующих алгоритмов определения заимствований в исходных кодах программ}

На основе~\cite{treb} была дана характеристика точности на небольшой объеме исходного кода. Все остальные характеристики оценивались на основе самых распространенных существующих алгоритмов из каждой группы методов.

В таблице \ref{tab:compare} приведена классификация ранее рассмотренных методов определения заимствований в исходных кодах программ. По горизонтали расположены критерии сравнения методов определения плагиата.
\begin{enumerate}
	\item Поддержка нескольких языков программирования.
	\item Определение типов заимствований.
	\begin{enumerate}
		\item[2.1)] Изменение комментариев и добавление пустых строк.
		\item[2.2)] Изменение имен функций, переменных и типов данных.
		\item[2.3)] Изменение порядка выполнения кода.
		\item[2.4)] Выделение частей кода в функции.
	\end{enumerate}
	\item Точность на небольшом объеме исходного кода. Находится как среднее из точности определения каждого из типа заимствований. 
	\item Низкоуровневое сравнение кода (ассемблерный листинг или байткод).
\end{enumerate}


\begin{table}[hbtp]
	\begin{center}
		\begin{flushleft}
			\caption{\label{tab:compare}Классификация существующих методов определения заимствований в исходных кодах программ}
		\end{flushleft}
		\begin{tabular}{|l | l | l | l | l | l | l | l |} 
			\hline 
             ~					& {1}	 &    {2.1} &   {2.2} & {2.3} & {2.4} & {3} & {4} \\ \hline
			Текст  & Нет  &   100\%  & 80\%    &  76\%   &  65\%    &  \textbf{80\% }& Нет  \\ \hline
			Токены  &  Да &   100\%  &  92\%   & 87\%  &  74\%  &    \textbf{88\% } & Нет  \\ \hline
		    Метрики &   Да &   100\%   &  87\%   & 95\%    &  72\%   & \textbf{89\% }& Нет  \\ \hline
			Деревья   &   Да &    100\%  &  93\%   &  91\%   &  50\%  &  \textbf{84\%}& Нет  \\ \hline
			Низкоуровневый код &  Да &    99\%  &  100\%   &   85\%   &   80\%  &  \textbf{91\%}& Да \\  \hline
		\end{tabular}
	\end{center}
\end{table}