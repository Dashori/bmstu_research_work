\section{Классификация существующих методов определения заимствований в исходных кодах программ}

На основе~\cite{treb} была дана характеристика точности на небольшой объеме исходного кода. Все остальные характеристики оценивались на основе самых распространенных существующих алгоритмов из каждой группы методов.

В таблице \ref{tab:compare} приведена классификация ранее рассмотренных методов определения заимствований в исходных кодах программ. По горизонтали расположены критерии сравнения методов определения плагиата:
\begin{itemize}[label*=---]
	\item 1 -- Поддержка нескольких языков программирования.
	\item 2.1 -- Определение первого типа заимствования. Изменение комментариев и добавление пустых строк.
	\item 2.2 -- Определение второго типа заимствования. Изменение имен функций, переменных и типов данных.
	\item 2.3 -- Определение третьего типа заимствования. Изменение порядка выполнения кода.
	\item 2.4 -- Определение четвертого типа заимствования. Выделение частей кода в функции.
	\item 3 -- Точность на небольшом объеме исходного кода. Находится как среднее из точности определения каждого из типа заимствований. 
	\item 4 -- Низкоуровневое сравнение кода (ассемблерный листинг или байткод).
\end{itemize}

\begin{table}[hbtp]
	\begin{center}
		\begin{flushleft}
			\caption{\label{tab:compare}Классификация существующих методов определения заимстований в исходных кодах программ}
		\end{flushleft}
		\begin{tabular}{|l | l | l | l | l | l | l | l |} 
			\hline 
             ~					& {1}	 &    {2.1} &   {2.2} & {2.3} & {2.4} & {3} & {4} \\ \hline
			Text-based   & Нет  &   100\%  & 80\%    &  76\%   &  65\%    &  \textbf{80\% }& Нет  \\ \hline
			Token-based  &  Да &   100\%  &  92\%   & 87\%  &  74\%  &    \textbf{88\% } & Нет  \\ \hline
			Metric-based &   Да &   100\%   &  87\%   & 95\%    &  72\%   & \textbf{89\% }& Нет  \\ \hline
			Tree-based   &   Да &    100\%  &  93\%   &  91\%   &  50\%  &  \textbf{84\%}& Нет  \\ \hline
			Binary-based &  Да &    99\%  &  100\%   &   85\%   &   80\%  &  \textbf{91\%}& Да \\  \hline
		\end{tabular}
	\end{center}
\end{table}