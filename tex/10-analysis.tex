\section{Анализ предметной области}

\subsection{Области применения определения заимствований в исходных кодах программ}
\subsubsection{Сфера образования}

Академический плагиат -- очень распространенное явление, рассмотрим его на примере университета. Иногда для определения заимствований не нужны специальные алгоритмы и автоматизированные системы, например:
\begin{itemize}[label*=---]
	\item исходный код проверяется преподавателем в присутствии студента и акт заимствования можно определить при разговоре;
	\item в образовательных целях студенту необходимо реализовать уже существующий алгоритм, например, конкретную сортировку массива целых  чисел. На такое задание заведомо есть верный ответ и решения студентов не будут сильно отличаться;
	\item количество студентов на курсе мало, преподаватель знает кто потенциально мог использовать чужой код и реализовывает проверку вручную.
\end{itemize}

Не всегда преподаватели считают, что плагиат это плохо, ведь иногда важнее понимание работы определенного кода, а не его реализация.

В остальных случаях, если преподаватели заинтересованы в выявлении плагиата, перед ними стоит задача найти оптимальный способ его определения, например --  автоматизация проверки исходных кодов.  

\subsubsection{Контесты по программированию}

На сегодняшний день существует множество различных соревнований по программированию, начиная от соревнования по самому запутанному коду на языке Си~\cite{ioccc} и заканчивая крупнейшей международной олимпиадой по программированию ICPC~\cite{icpc}. Также контесты иногда проводятся при отборе на работу в крупные компании.

Плагиат в таких случаях часто играет решающую роль, ведь только на его основе можно определить уровень участника.

\subsubsection{Разработка программного обеспечения}
Плагиат при разработке программного обеспечения встречается намного реже, так как существует юридический документ, определяющий  его использование и распространение -- лицензия на программное обеспечение. Такие лицензии делятся на два типа: несвободные (исходный код закрыт) и свободные (открытое программное обеспечение). Их различия сильно влияют на права конечного пользователя в отношении использования программы.

\subsection{Критерии принадлежности кода к плагиату}
Опираясь на~\cite{treb} были выделены 4 основных типа заимствования исходного кода.
 \begin{enumerate}
 	\item Программный код скопирован без каких-либо изменений (идентичен оригиналу с точностью до комментариев).
 	\item Код скопирован с <<косметическими>> заменами идентификаторов (имен функций и переменных, типов данных, строковых литералов).
 	\item Код может включать заимствования второго типа и модифицирован путем добавления, редактирование или удаления его фрагментов или бесполезных участков кода. Также возможны изменения порядка, не влияющие на логику самой программы.
 	\item Программа некоторым образом переписана с общим сохранением логики работы и функциональности, однако синтаксически она может абсолютно отличаться от оригинала.
 \end{enumerate}

Заимствования четвертого типа крайне затруднительны для выявления, зачастую это приводит к нахождению плагиата алгоритмов, а не исходного кода. Так, например, в приведенном выше примере со студентами и сортировкой массива целых чисел, плагиатом будут считаться только заимствования первого типа.

\subsection{Критерии сравнения методов определения плагиата}
На основе приведенного выше анализа предметной области были выделены следующие критерии сравнения методов определения плагиата.
\begin{enumerate}
	\item Поддержка нескольких языков программирования.
	\item Определение каждого из типов заимствования.
	\item Точность на небольшом объеме исходного кода. 
	\item Низкоуровневое сравнение кода (ассемблерный листинг или байткод).
\end{enumerate}

Первый критерий важен в ситуациях, когда работы, которые необходимо сравнить, написаны на разных языках программирования. Определение каждого из типов заимствований важно, так как на этой основе можно оценить уровень заимствования программного кода. Критерий <<точность на небольшом объеме исходного кода>> был выбран из-за специфики сферы образования и контестов по программированию (как правило размер кода в таких задач не является большим). Низкоуровневое сравнение кода было выделено, потому что в настоящее время это новое и очень перспективное направление в сфере определения заимствований.
\pagebreak