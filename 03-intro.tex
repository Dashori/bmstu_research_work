\section*{ВВЕДЕНИЕ}
\addcontentsline{toc}{section}{ВВЕДЕНИЕ}

Заимствование материала является распространенной проблемой как в сфере образования, так и в сферах научной и культурной деятельности. Зачастую это может повлечь за собой нарушение авторско-правового и патентного законодательства~\cite{law}. Соответственно, возникает необходимость установления авторских прав на результаты интеллектуального труда.

Основная область применимости проверки на заимствования исходных кодов программ -- сфера образования, а также соревнования по программированию. Это не только нарушает авторские права, но и отрицательно сказывается на качестве образования.

Цель работы -- провести обзор существующих методов определения заимствований в исходных кодах программ.
Для достижения этой цели требуется решить следующие задачи:
\begin{itemize}[label*=---]
	\item провести анализ предметной области и обзор существующих решений;
	\item установить критерии принадлежности программного кода к категории плагиата;
	\item сформулировать критерии сравнения алгоритмов;
	\item классифицировать существующие методы определения заимствований в исходных кодах программ.
\end{itemize}

\pagebreak